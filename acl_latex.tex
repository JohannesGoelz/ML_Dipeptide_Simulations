% This must be in the first 5 lines to tell arXiv to use pdfLaTeX, which is strongly recommended.
\pdfoutput=1
% In particular, the hyperref package requires pdfLaTeX in order to break URLs across lines.

\documentclass[11pt]{article}

% Remove the "review" option to generate the final version.
\usepackage[final]{acl}

\usepackage{nameref}
\usepackage{amsmath}

% \usepackage[vmargin=1in,hmargin=1in]{geometry}
\usepackage{enumitem}
\setlist[itemize]{topsep=0pt,before=\leavevmode\vspace{-1.5em}}
\setlist[description]{style=nextline}

\usepackage{tikz}
\usetikzlibrary{positioning}
\usetikzlibrary{calc}
\usepackage{textcomp}
\usepackage[T1]{fontenc}
\usepackage{tipa}

% Standard package includes
\usepackage{times}
\usepackage{latexsym}
\usepackage{comment}

% For proper rendering and hyphenation of words containing Latin characters (including in bib files)
\usepackage[T1]{fontenc}
% For Vietnamese characters
% \usepackage[T5]{fontenc}
% See https://www.latex-project.org/help/documentation/encguide.pdf for other character sets

% This assumes your files are encoded as UTF8
\usepackage[utf8]{inputenc}

% This is not strictly necessary, and may be commented out,
% but it will improve the layout of the manuscript,
% and will typically save some space.
\usepackage{microtype}


\usepackage[acronym]{glossaries}

\makeglossaries

\newacronym{ml}{ML}{Machine Learning}
\newacronym{cg}{CG}{Coarse Graining}
\newacronym{nn}{NN}{Neural Network}
\newacronym{gb}{GB}{Generalized Born \cite{still1990semianalytical, Onufriev2019-yz}}
\newacronym{md}{MD}{Molecular Dynamics}
\newacronym{gnn}{GNN}{Graph Neural Network}
\newacronym{remd}{REMD}{Replica-Exchange Molecular Dynamics}
\newacronym{mlp}{MLP}{Multilayer Perceptron}
\newacronym{qm}{QM}{Quantum Mechanics}
\newacronym{pmf}{PMF}{Potential of Mean Force}
\newacronym{kl}{KL}{Kullback–Leibler}
\newacronym{ase}{ASE}{Atomic Simulation Environment}
\newacronym{ffn}{FFN}{Feedforward Neural Network}

% If the title and author information does not fit in the area allocated, uncomment the following
%
%\setlength\titlebox{<dim>}
%
% and set <dim> to something 5cm or larger.

\title{Automatic Speech Recognition}

% Author information can be set in various styles:
% For several authors from the same institution:
% \author{Author 1 \and ... \and Author n \\
%         Address line \\ ... \\ Address line}
% if the names do not fit well on one line use
%         Author 1 \\ {\bf Author 2} \\ ... \\ {\bf Author n} \\
% For authors from different institutions:
% \author{Author 1 \\ Address line \\  ... \\ Address line
%         \And  ... \And
%         Author n \\ Address line \\ ... \\ Address line}
% To start a seperate ``row'' of authors use \AND, as in
% \author{Author 1 \\ Address line \\  ... \\ Address line
%         \AND
%         Author 2 \\ Address line \\ ... \\ Address line \And
%         Author 3 \\ Address line \\ ... \\ Address line}

% \author{Vorname Nachname \\
%   Ihre Einrichtung / Firma \\
%   \texttt{ihre_email@adresse.com}}

\author{Johannes Gölz \and Paul Link\\
  Karlsruhe Institute of Technologie \\
  \texttt{johannes.goelz2@student.kit.edu} \\}

\begin{document}
\maketitle

\begin{abstract}

The abstract is

\end{abstract}


\section{Introduction}
\label{sec:introduction}

\begin{comment}
Draft:
- Motivation for the subject
- Why current methods are not sufficient
- Image of molecule in waterbox
- Sota
- Transferability
- Image of transferability
- Our scope
- The goal
\end{comment}

% Motivation
Understanding molecular behavior at the atomistic scale is essential for applications such as drug discovery and materials design. Designing a molecule for a specific function, for example, binding to a target protein or maximizing light absorption in photovoltaic devices, requires knowledge of how that molecule adopts different conformations in its target environment. This, in turn, necessitates sampling from the Boltzmann distribution, which quantifies the probability of observing a given molecular conformation.


% Drawbacks of current methods
In this context, solvent effects play a decisive role. In biomolecular systems, for instance, water is the predominant medium and strongly influences conformational sampling through hydrogen bonding, dielectric screening, and hydrophobic interactions. More generally, the inclusion of solvents can yield Boltzmann distributions that differ significantly from those obtained in vacuum settings, underscoring the importance of accurately accounting for solvation effects across diverse domains. 

However, simulating a single target molecule, such as a drug candidate, in an explicit water box containing thousands of solvent molecules incurs substantial computational cost. Classical implicit solvent models, such as the \gls{gb} approximation, mitigate this by eliminating explicit solvent degrees of freedom. While computationally efficient, these models often sacrifice accuracy relative to fully atomistic descriptions.


\begin{figure}
    \centering
    \includegraphics[width=1\linewidth]{images/solvated_alanine.png}
    \caption{Dieses Bild zeigt das Miniprotein alanine-dipeptid in Wasser. Zu dem Molekül geören 22 Atome. Die Wasserbox ist quadratisch und hat eine seitenlänge von 2nm und enthält 696 Atome, welche zum Wasser gehören}
    \label{fig:solvated_alanine}
\end{figure}


% Short related Work / Sota
% much shorter then in Proposal
Recent progress in \gls{ml} has opened new avenues for developing more accurate and transferable implicit solvation models. A pioneering ML-based approach was proposed by \citet{machine_learning_implicit_solvation} with ISSNet, which demonstrated the feasibility of learning solvation effects directly. While their study did not address the transferability of the model, subsequent work has made progress toward more generalizable implicit solvation models \cite{katzberger2025rapid, Airas2023, röcken2025predictingsolvationfreeenergies}, although several key challenges remain unresolved.


A critical analysis of the current literature on transferable implicit solvation models reveals a significant research gap, particularly concerning their application to macromolecules like proteins. The state-of-the-art can be effectively divided into two domains: models developed for small organic molecules and those targeting proteins.

% Transferatbiity



In summary, while implicit solvation models for small drug like molecules are maturing, some approaches may face scalability challenges. For proteins and peptides on the other hand, the central unresolved issue is transferability, as existing models are data-scarce and fail to reliably generalize to new sequences or protein families. From this we derive our central research question:  

\begin{itemize}
    \item How can we develop a \gls{ml} implicit solvation model that generalizes reliably across diverse peptide and protein sequences as well as structures?
\end{itemize}

We hypothesize that the inherent modularity of peptides and proteins, which are composed of a limited set of 20 canonical amino acids, provides a solid basis for a \gls{ml} model to learn the fundamental interactions governing solvation from a computationally tractable dataset of small peptides. Furthermore, the combinatorial nature of the peptide domain allows for a systematic evaluation of the implicit solvent model with respect to its transferability to unseen sequences or larger, previously unseen polypeptide chains. Later, we can additionally include drug-like molecules, such that protein-ligand interactions can be accurately modeled.

\section{Related Work}
\label{sec:related_work}

\citet{machine_learning_implicit_solvation} were the first to apply \gls{ml} methods to implicit solvation. Their approach employed a \gls{gnn} trained with a force-matching objective to learn solvent effects from explicit solvent simulation data. Several variants of the model were developed, differing in the type of ground-truth information employed. Specifically, the feature vectors were derived from either atom types, partial charges, or a mixture of both. Compared to the baseline \gls{gb} model, their method achieved lower \gls{kl} divergence on molecules such as chignolin and alanine dipeptide. However, the model lacked transferability across systems. It could only reproduce the Boltzmann distribution of a system for which sufficient training data were already available. While this work laid important groundwork in the field, its limited generalizability restricted its practical applicability.

Recent developments have looked forward to refine implicit solvation through machine learning. One approach is delta learning \cite{katzberger2024general, katzberger2025rapid}, which builds on classical solvation models by training a model to predict the difference between implicit and explicit solvent results. In practice, a \gls{nn} improves accuracy by modifying parameters within a \gls{gb} baseline, such as scaling the Born radii, while preserving computational efficiency. \citet{katzberger2024general} extended the model to multiple solvents, achieving speedups of about one order of magnitude compared to explicit-solvent simulations while reproducing free-energy profiles of small organic molecules. In particular, the method can recover \gls{pmf} curves of intramolecular hydrogen bonds across diverse solvents. However, the model has not been tested on molecules weighing more than 700 Da.

Another relevant approach is potential contrasting, introduced by \citet{Airas2023}. In this method, a pretrained implicit-solvation SchNet potential is refined by contrasting conformations obtained from explicit-solvent simulations with those from the implicit baseline. The model learns to discriminate realistic samples from noise-like ones, thereby eliminating the need for labeled force data. The authors trained their model on six proteins containing between 166 and 624 atoms and reported that it outperformed the GB baseline on these systems. To assess transferability, they evaluated two mutations of the protein chignolin, which was part of the training set. For the mutation with 80\% sequence identity, the model achieved reasonable accuracy, whereas for the mutation with 70\% sequence identity, the GB baseline performed better. Additionally, the authors tested the model on the intrinsically disordered peptide JIP1, which has been extensively studied by \citet{jipidp}. In this case, the SchNet model failed to capture the main energy basin observed in explicit-solvent simulations, which the GB baseline successfully reproduced. Nevertheless, the SchNet model identified three of the remaining basins, albeit with differing relative energy levels.

A further line of work was introduced by \citet{röcken2025predictingsolvationfreeenergies}, who combined vacuum pre-training with solvent refinement for small, drug-like molecules. In this approach, their model ReSolv first learns a vacuum potential from \gls{qm} data using the QM7-X dataset \cite{Hoja2021QM7X} and subsequently refines it using experimental solvation free energies from the FreeSolv dataset \cite{DuarteRamosMatos2017FreeSolv}. For training, the authors identified 559 overlapping molecules (each containing up to 23 atoms) between the two databases and split them into 70\% training and 30\% test sets. The split was carefully designed to maintain consistent heavy-atom combinations and to ensure that no unseen chemical functional groups appeared in the test set. Their model consistently outperformed the CHARMM \cite{mackerell1998all} and AMBER \cite{cornell1995second} force fields with explicit water. Furthermore, they demonstrated partial transferability across chemical functional groups by retraining on modified datasets in which one functional group was omitted. Nevertheless, the observed increase in error for molecules with more than 14 heavy atoms suggests potential scaling limitations. Moreover, the primary results presented in the paper focus on in-distribution generalization due to the engineered dataset split, leaving questions about true out-of-distribution performance. Finally, a detailed analysis of whether the \gls{ml} potential can reproduce correct ensembles of molecular conformations is absent. Such an analysis is inherently important for applications like drug discovery or protein folding, where accurately capturing the Boltzmann distribution is as critical as predicting solvation free energies.



\section{Data Generation}
\label{sec:data_generation}

A central component of the project is the generation of large-scale simulation data that can serve as training input for \gls{ml} models. In these simulations, solutes are solvated in a water box, equilibrated, and then propagated over time to obtain representative trajectories. Both coordinates and forces are stored, which allows the resulting datasets to be used in diverse training settings such as force-matching.

To improve sampling efficiency, \gls{remd} \cite{qi2018replica} can be incorporated. By running simulations at multiple temperatures and exchanging configurations, \gls{remd} helps to explore conformational space more thoroughly, especially for flexible solutes. This is important in order to avoid biased datasets that overrepresent only a limited set of conformations.

The overall data pipeline consists of several stages. First, solute structures are prepared and solvated in explicit water boxes. Next, energy minimization and equilibration steps ensure physically meaningful starting points. The Simulations then generate snapshots of the coordinates and forces, which are downsampled and stored in standardized formats. 

The precise scope of data generation in this project is to assemble datasets for model development and benchmarking. While the current focus lies on explicit water with standard force fields, the approach can naturally be extended to other solvents in later stages.


\section{Implicit Solvation Model}
\label{sec:implicit_solvation_model}

The Model is


\section{Conclusion}
\label{sec:conclusion}

The Conclusion is

% Entries for the entire Anthology, followed by custom entries
\bibliography{anthology,custom}

\appendix

\printglossary[type=\acronymtype]

\end{document}
